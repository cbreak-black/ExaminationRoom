\subsection{Used Frameworks\label{frameworks}}
\paragraph{}
When developing an application, it is impossible to write everything from scratch.
Frameworks and libraries help developers not spending time on reinventing wheels,
but on finding solutions to the specific problems of their program.

\ER\ uses several libraries for window management, memory management, scripting and rendering.


\subsubsection{Qt\label{FrameworkQt}}
\paragraph{}
A big part of the functionality comes from the operating system and associated platform APIs.
This includes basic necessities like reading data from a file, drawing data to screen or specialized functionality like drawing a tree view, scaling images or managing tool windows.

\paragraph{}
All operating systems offer their own API to do some of those things.
\textit{POSIX} is supported on Linux, Mac OS X and Windows, but it only covers basic file IO, and nothing graphical.
Specific APIs like \textit{Mac OS X'}s \textit{Cocoa} or Windows' \textit{.Net} have a much richer feature set, but limit the application to that platform.
This might be acceptable for widely distributed end user applications where a fraction of potential users can be ignored,
but in a research environment the software has to be able to use existing hardware optimally.
A solution is \textit{Qt}\cite{qt}:

\begin{quotation}
Qt is a cross-platform application framework for desktop and embedded development. It includes an intuitive API and a rich C++ class library, integrated tools for GUI development and internationalization, and support for Java™ and C++ development.
\end{quotation}

\paragraph{}
An alternative would be to use \textit{Java} with it's associated APIs, but it's support for hardware accelerated drawing is lacking.


\subsubsection{Boost\label{FrameworkBoost}}
\paragraph{}
\textit{C++} is an old language, and has it's root in the language \textit{C}.
Neither has advanced memory management tools, be it a garbage collector like \textit{LUA} or \textit{Java}, or a reference counting system like used in \textit{Objective-C}.
Managing memory manually is a challenge, doing it wrong can cause crashes that are hard to debug.

A solution to this problem is offered by \textit{TR1}, an extension to the C++ standard library.
Boost\cite{boost} is a provider of a \textit{TR1} implementation.

\paragraph{}
\textit{TR1} contains among others a robust and flexible implementation of a set of shared-ownership smart pointers.
As long as a smart pointer to a memory location exists, it can be used like a normal pointer thanks to \textit{C++}'s operator overloading,
but as soon as all pointers to it go out of scope, the memory will be automatically deallocated.

\paragraph{}
An other useful feature in boost are function binders.
while standard \textit{C} functions and static functions can be pointed at by a simple pointer,
\textit{C++} class member functions require more data
(Method Pointers vary in size depending on compiler and platform, and can be 12 bytes or more).
Together with the strict type checking of \textit{C++}, this makes storing pointers to member functions complicated and inflexible.

With \textit{TR1}'s functional library, binding any kind of method or function and storing it is easily possible, while still retaining strict type checking.
This is used to implement callbacks.


\subsubsection{OpenGL\label{FrameworkOpenGL}}
\paragraph{}
Since the whole goal of the tool is to generate stereoscopic stimuli and present them to test subjects,
the graphical rendering framework is of tremendous importance.
In the beginning of graphical computers, programs drew everything on their own, and accessed the frame buffer directly.
In modern systems, drawing is highly abstracted and often hardware accelerated.

\ER\ requires the ability to draw 3D scenes and reproduce many effects such as lighting, cameras, atmospheric effects and other depth cues.
The goal is to be able to test both abstract scenes consisting only of few shapes and a limited amount of cues,
as well as scenes that contain many cues, and more closely approximate real visuals.
This is similar to what computer games use.

\paragraph{}
The only cross platform 3D drawing toolkit is \textit{OpenGL}\cite{opengl}, a standard developed by \textit{Silicon Graphics} in 1992.
It's a state-driven procedural C API that was extended with many additions to the standard since it's inception.
Many features of \textit{OpenGL} are accelerated by modern graphic cards, and allow real time animation rendering.

The \ER\ renderer uses \textit{OpenGL} for the scene output.
Objects paint them-selfs with OpenGL commands.
The offerings of the framework are not perfect, but it allowed to realize almost all desired features.
A more advanced anaglyph renderer would likely be possible to do with frame buffer objects and pixel shader, but that would exclude a wide range of older installations and hardware.


\subsubsection{LUA\label{FrameworkLua}}
\paragraph{}
It was clear from the beginning of the project that the only way to get the required flexibility to do anything, and test anything can not be achieved by simply defining a scene format and some animation paths.
Tests like a flight simulator for motion sickness require interactivity in complex ways.
The chosen solution was to use a programming language to specify interaction and mechanics of a test.
The design of a language is complex, and many scripting languages are already established, so it was decided to reuse an existing language.

Most scripting languages are either intended to be used in stand-alone scripts such as \textit{Perl} or various shell scripting languages.
\textit{Python} can be used as  embedded language, and offers a very rich API on it's own.
The downside is that it is huge and comes with many features that are not needed, which complicate
the integration into custom applications.

\paragraph{}
The chosen language is \textit{LUA}\cite{lua}, a small library designed to be used as embedded language.
It provides all basic features, comes with garbage collection, flexible syntax and powerful binding API.
It's syntax is easy to understand, but still offers enough sugar to use not only procedural, but also modular, object oriented or functional programming.
It's license is very liberal, which made it popular even in the comercial game industry.
The official project description emphasizes those points.

\begin{quotation}
Lua is an extension programming language designed to support general procedural programming with data description facilities. It also offers good support for object-oriented programming, functional programming, and data-driven programming. Lua is intended to be used as a powerful, light-weight scripting language for any program that needs one. Lua is implemented as a library, written in clean C (that is, in the common subset of ANSI C and C++).
\end{quotation}

\paragraph{LuaBridge}
While the \textit{LUA} C API is very powerful, it is also very low level.
Writing code that interacts with \textit{LUA} and especially code that is called by it requires following the calling convention of \textit{LUA}.
Manually handling the low level API helped with understanding the language better and finding problems,
but it also required to maintain two interfaces, one that exposes functions to \textit{C++} and a second set for \textit{LUA}.

\textit{LuaBridge} automatically generates this binding code in a very elegant way.
Unlike tools like \textit{tolua++} it does not require a preprocessor,
it uses template meta programming to generate wrapper code.
\textit{LuaBind} works similar, but \textit{LuaBridge} is smaller, easier to understand and has less dependencies.

