\subsection{Used Frameworks\label{frameworks}}
\paragraph{}
When developing an application, it is impossible to write everything from scratch.
Frameworks and libraries help developers not spending time on reinventing wheels,
but on finding solutions to the specific problems of their program.

\ER\ uses several libraries for window management, memory management, scripting and rendering.


\subsubsection{Qt}
\paragraph{}
A big part of the functionality comes from the operating system and associated platform APIs.
This includes basic necessities like reading data from a file, drawing data to screen or specialized functionality like drawing a tree view, scaling images or managing tool windows.

\paragraph{}
All operating systems offer their own API to do some of those things.
\textit{POSIX} is supported on Linux, Mac OS X and Windows, but it only covers basic file IO, and nothing graphical.
Specific APIs like \textit{Mac OS X'}s \textit{Cocoa} or Windows' \textit{.Net} have a much richer feature set, but limit the application to that platform.
This might be acceptable for widely distributed end user applications where a fraction of potential users can be ignored,
but in a research environment the software has to be able to use existing hardware optimally.
A solution is \textit{Qt}\cite{qt}:

\begin{quotation}
Qt is a cross-platform application framework for desktop and embedded development. It includes an intuitive API and a rich C++ class library, integrated tools for GUI development and internationalization, and support for Java™ and C++ development.
\end{quotation}

\paragraph{}
An alternative would be to use \textit{Java} with it's associated APIs, but it's support for hardware accelerated drawing is lacking.


\subsubsection{Boost}
\paragraph{}
\textit{C++} is an old language, and has it's root on the language \textit{C}.
Neither has advanced memory management tools, be it a garbage collector like \textit{LUA} or \textit{Java}, or a reference counting system like used in \textit{Objective-C}.
Managing memory manually is a challenge, doing it wrong can cause crashes that are hard to debug.

A solution to this problem is offered by \textit{TR1}, an extension to the C++ standard library.
Boost\cite{boost} is a provider of a \textit{TR1} implementation.

\paragraph{}
\textit{TR1} contains among others a robust and flexible implementation of a set of shared-ownership smart pointers.
As long as a smart pointer to a memory location exists, it can be used like a normal pointer thanks to \textit{C++}'s operator overloading,
but as soon as all pointers to it go out of scope, the memory will be automatically deallocated.

\paragraph{}
An other useful feature in boost are function binders.
while standard \textit{C} functions and static functions can be pointed at by a simple pointer,
\textit{C++} class member functions require more data.
Together with the strict type checking of \textit{C++}, this makes storing pointers to member functions complicated and inflexible.

With \textit{TR1}'s functional library, binding any kind of method or function and storing it is easily possible, while still retaining strict type checking.
This is used to implement callbacks.


\subsubsection{OpenGL}
\paragraph{}
\todo{Compare to other drawing methods}

\subsubsection{LUA}
\paragraph{}
\todo{Compare to other scripting}

\paragraph{LuaBridge}

