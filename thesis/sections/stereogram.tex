\section{Stereogram}
\paragraph{}
A stereogram is an image that contains separate data for the left and right eye. Specially crafted images can cause a binocular depth perception. The framework supports stereograms in various formats. The simplest case consists of two images, one for each eye. Several types of random dot stereograms are also available.

\subsection{Random dot stereogram}
\paragraph{}
Random dot stereograms are a topic that has interested scientists since over a century\cite{AntRDS}. In the early 1960s they where introduced as stimuli into the modern neuroscience by Julesz\cite{BellRDS}. Unlike a normal stereogram, a RDS contains no monocular clues of the depth, or any clue about the stereoscopic image at all. This makes them the perfect tool to study binocular vision.

The two images of an RDS consist of a random pattern, usually dots. Since either image on it's own is purely random, no information can be derived of it. When seen with binocular vision, depth can be seen. There are several types of random dot stereograms, and their principle is the same: Depth information is used to shift parts of the pattern in the texture on the horizontal axis.