\section{Example Tests\label{Example}}
\paragraph{}
The best way to learn something is seeing how it's done and understanding why it is done.
This section gives a few examples for test scenes.

\subsection{A small scene}
\paragraph{}
The source code for this test is bundled with the \ER\ distribution.
It can also be found in the appendix in listing \ref{example.small.lua}.

This example shows how to create the skeleton of a test, including animation and key input.

\subsubsection{Data}
\paragraph{}
At the start of the scene come the settings.
Camera position, viewing properties and other settings are declared here so they can easily be changed.

Tables containing texture paths are declared after line 28.
Note that the use of tables allows randomisation by permuting the table, an easy procedure.
Similarly the possible answers are stored in a table.

Starting on line 45, the key mappings are declared.
Key event handlers receive a char representing the pressed key, creating a table that maps expected input chars to strings makes the code much more readable.

\subsubsection{Visual}
\paragraph{}
The visual properties of the scene are set up from line 55 onwards.
First the external aspects are written to the log.
This is good practice and helps keeping the log files sorted.

Then the camera parameters are set to the values previously declared.
The values are logged as well.

The objects seen in the scene are created and assigned to variables.
It is not needed to keep references to objects that aren't intended to be changed again.
The first object in this scene is a rectangle, so the two direction vectors, the position and the texture is set and the rectangle is added to the scene.

Next the targets are created.
For a variable number of targets, using a list to store them simplifies access.
The pixel-planes for the target get created and added to the list and scene.

There are no rules how to build a scene graph, but it is important to remember that objects are reference types.
An object can only be in one place at a time.
It is removed from its previous place when it is inserted somewhere else.

\subsubsection{Mechanic}
\paragraph{}
The animation and reaction part of the test can be seen from line 106.
A common helper function is \lstinline[language=lua]{permuteTable()}, which permutes a table randomly.
In many tests random permutations are used.

Most tests follow the same pattern: A stimulus is displayed, then user input is received.
To display the next stimulus, the method \lstinline[language=lua]{displayTarget()} is used.
It randomly selects a shape from the shape table, a direction from the direction table and the position.
The new texture for the target pixel-plane is created and added to the texture.
The commented-out code shows alternative texture types.
In a real test, this function would also log the new parameters to the log file, and possibly modify other parameters of the scene graph.

\paragraph{}
To handle user input, a keyDown event handler is registered.
In this example the function \lstinline[language=lua]{parseInput()} is this event handler.
The general structure of key event handler is the same in most cases, first the key is checked against the table defined earlier.
If it's not contained in the table a warning is logged and the key is ignored.
Keys that are recognized are acted on by checking against the expected reply and progressing the scene, or as in this example by directly changing the scene.

Animations are implemented by registering an update handler.
The update event is sent between each frame, the interval from the last call is passed as parameter.
This example has a simple formula-based animation path for the second target, the update handler calculates the new position and directly changes it.


\subsection[Influence of continuous depth]{Pilot Study: Influence of continuous depth on fatigue\label{ExampleFatiguePilot}}
\paragraph{}
This experiment is described in detail in section \ref{FatiguePilot}.
Here, the implementation details are described, and parts of the source code is explained.
Note that the code used here is not the one used in the experiment, it was cleaned up to be more beneficial as example.
In addition to the techniques displayed in the last section, this test also uses the questionnaire library and the use of the GUI designer.

\subsubsection{Visual}
Visually this test is simple, a textured floor plane and one or two pixel planes dominate the scene.
The floor is textured with a checkerboard pattern, the pixel planes show a random dot encoded depth image.
To emphasize the object of interest, a yellow rectangle outline is used.

This scene is created in the GUI.
Start a new project, and add two pixel planes with name \texttt{pp0} and \texttt{pp1}.
The texture of the targets will get created dynamically, so it doesn't have to be assigned.
The same is true for their position.
Disable automatic rescaling for the pixel planes, the test targets should keep their size independently from their camera distance.

Create a yellow wireframe rectangle as marker for the active target.
Just as the target planes, its position and size is dynamically calculated.
Reduce the number of subdivisions to 0, so only the outline is shown.

Finally, add a rectangle as floor plane.
Give it a simple checkerboard texture and set its vectors to point into the x and z directions.
Depending on the chosen vector length, set the texture coordinates so that the texture is mapped correctly and with the desired number of repetitions.

To add code, switch to the code panel.
Add the existing library files statistics.lua and questions.lua, which are used to calculate some statistics, and to conduct a questionnaire.
Add a new file to contain your own code.
Then save the scene to a new file.
This is the root file referring all media and code files, It should not be edited manually since it gets rewritten every time the scene is exported.

\subsubsection{Mechanics}
\paragraph{}
The source code to the algorithms described here can be found in \texttt{example.cycle.code.lua}.
The core of the scene mechanic is a progression function.
Many helper functions assist with the progression and the randomisation.

The first part of the code file consists of constant declarations.
Some are test settings, and define the look of the texture, the audio feedback or the length of the parts of the scene.
Others describe the external factors of the test and are used for statistical calculations.

The randomisation is balanced in a very special way:
In a cycle with 18 tests, each depth change appears exactly twice, once from left to right, and once from right to left.
A depth change is the change between the old and new target in depth.
To easily get this type of random order, six such random orders are hard-coded on line 35.
The positions those random orders refer to are permuted, making from each random cycle 36 versions.
An alternative implementation would brute force randomize cycles.

The positions are defined as list of possible values per coordinate.
In the final test, just one shape was used as depth map.
Additional shapes could be added to the list \lstinline[language=lua]{shapes} and would be randomly selected.

From line 56 onwards, as in the previous example, a key code lookup table translates the char codes into strings.
An input lookup table translates the key strings into meaningful strings for this test.

The statistics library needs to be initialized to be used.
This is done on line 70, followed by logging the scene parameters.
Having the exact properties of the test in the log file helps with analyzing.

The objects created in the GUI can be accessed with their name.
For us it would however be more comfortable to store them in a list, so we just do that.

\begin{lstlisting}[language=lua,firstnumber=83]
-- Make targets easier accessible
targets = {
	pp0,
	pp1
};
\end{lstlisting}

Next, the state is set up, it defines the current test status and the immediate future tests.
\texttt{cycleNum} and \texttt{blockNum}  define the current position within the test.
\texttt{currentBlockType} defines the type of the block (if the floor is shown or not).
\texttt{currentCycle} point to the currently selected random cycle described above.
\texttt{currentSide} states on which side the current target lies, \texttt{currentTest} keeps the number of the current test stimulus.
Finally, \texttt{labelToIndex} stores the translation tables between the random cycle and the position in depth.

\begin{lstlisting}[language=lua,firstnumber=89]
-- State for progression function
cycleNum = 0;
blockNum = 0;
currentBlockType = 1; -- 1 = continuous depth, 2 = no continuous depth
currentCycle = math.random(1, #perfectCycles);
currentSide = math.random(1, 2); -- 1 = left, 2 = right, as defined in targetX
currentTest = 0;
labelToIndex = {
	{ 1, 2, 3}; -- first side (even)
	{ 1, 2, 3}; -- second side (odd)
}
replies = {};
\end{lstlisting}

On line 103 the test is prepared.
To balance the replies, so that both possible replies (convex and concave) are approximately equal in number, a table is filled with the same number of both replies.
This replies table is later permuted randomly to get the expected replies for a cycle.

Lookup functions help with getting the position for a given test stimulus in the current cycle.
Those methods take the random permutation of depth into account.

\paragraph{Progression}
The \texttt{nextBlock}, \texttt{nextCycle} and \texttt{nextTarget} methods are responsible to set up the next test.
\texttt{nextTarget()} (line 161) is called after every valid reply.
If the current cycle has ended, that is the next test would be outside the pre-computed cycle, the next cycle is started by calling \texttt{nextCycle()} (line 147).
If the current block has ended, this function also goes to the next block by calling \texttt{nextBlock()} (line 137).
The new test stimulus is then logged and displayed.

An easier test with less dependencies does not require that much state, but the need for non overlapping stimuli requires some book-keeping.
It is beneficial to separate the progression functions for different levels of the test, generally blocks and stimuli.

\paragraph{Displaying}
To displaying the newly calculated test stimulus, the parameter of objects created in the GUI are changed.
First the parameters for the current test are found from the state.
Then the new texture is generated using the parameters defined at the beginning of the file:

\begin{lstlisting}[language=lua,firstnumber=202]
local texture = RandomDot(shape);
texture:setMaxColor(maxColor);
texture:setExclusiveColor(exclusiveColor);
texture:setStyle(replies[currentTest]); -- Here the concave/convex status is set
\end{lstlisting}

To keep the target at a constant size, it is moved to the center, resized to the size it should have there, and then moved to the new position.
This is pixelplane specific code which is not needed by default, since auto-resize usually takes care of the correct size.

The marker is a rectangle, and therefore unlike pixel-planes gets projected by the camera transformation.
To make it appear to be of the same size as the target, it is scaled with the camera scale factor between the center and the target point.
The position of the marker is slightly in front of the target.
The floor is shown or hidden depending on the current block type.
As last step the parameters are logged.

As in the previous test, the \texttt{parseInput} function handles key events.
It tests if the given input is valid (an allowed key in the lookup table), and then if the input is correct.
Correctness is validated with the current state and of course logged.
Audio feedback is played in some cases, depending on settings.
Voice output only works on \textit{Mac OS X}, a shell is opened to execute the command line utility \texttt{say}.

\begin{lstlisting}[language=lua,firstnumber=202]
Scene:log("Input Correct: "..d);
if (audioCorrect) then
	Scene:beep();
end
if (voiceCorrect) then
	os.execute("say -v "..voice.." correct &");
end
\end{lstlisting}


\paragraph{Questionnaire}
The questionnaire library handles questionnaires.
Two types are supported:
Graded questions demand input in a range of specified granularity.
Multiple-choice questions demand as input one of the offered choices.

Beginning on line 301 the questionnaire is set up.
A table with all questions is built by using the function \lstinline[language=lua]{questions:createQuestion()}.
In its first form it takes as argument a question, the minimal and maximal grade label and the number of grades.
In its second form it takes a question as first argument, followed by a list of possible answers.

The function \lstinline[language=lua]{startQuestions()} starts the questionnaire with the passed callback.
Callbacks are very important, they allow the question library to return control to the invoking code once all questions are asked.
Three callbacks are used, one for the first questionnaire to set up initial conditions and start the test, one for the center tests to restore conditions and continue the test, and one for the last questionnaire to exit the scene.
It is important to remember that the questionnaire registers its own event handlers, so all handlers have to be installed after the library returned.

