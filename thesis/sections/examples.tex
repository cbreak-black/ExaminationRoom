\section{Example Tests}
\paragraph{}
The best way to learn something is seeing how it's done and understanding why it is done.
This section gives a few examples for test scenes.

\subsection{A small scene}
\paragraph{}
The source code for this test is bundled with the \ER\ distribution.
It can also be found in the appendix in listing \ref{example.small.lua}.
\todo{Add the small scene description}


\subsection[Influence of continuous depth]{Pilot Study: Influence of continuous depth on fatigue\label{ExampleFatiguePilot}}
\paragraph{}
This experiment is described in detail in section \ref{FatiguePilot}.
Here, the implementation details are described, and parts of the source code is explained.
Note that the code used here is not the one used in the experiment, it was cleaned up to be more beneficial as example.
In addition to the techniques displayed in the last section, this test also uses the questionnaire library and the use of the GUI designer.

\subsubsection{Visual}
Visually this test is simple, a textured floor plane and one or two pixel planes dominate the scene.
The floor is textured with a checkerboard pattern, the pixel planes show a random dot encoded depth image.
To emphasize the object of interest, a yellow rectangle outline is used.

This scene is created in the GUI.
Start a new project, and add two pixel planes with name \texttt{pp0} and \texttt{pp1}.
The texture of the targets will get created dynamically, so it doesn't have to be assigned.
\todo{Explain how to design that in the GUI designer}

\subsubsection{Mechanics}
\paragraph{}
The core of the scene mechanic is a simple progression function.
\todo{Explain how it works}
\todo{Describe the connection of test, cycle, block}
\todo{Explain why the cycles are randomized like they are}


\paragraph{Questionnaire}
\todo{Write how they are used}

