\subsection{User Interface\label{ImplementationUI}}
\paragraph{}
The User Interface is implemented with \textit{Qt}'s extensive widget classes.
\textit{Qt GUI} aims to provide native looking controls on all platform.
An experienced user can tell the difference between native applications and one written with \textit{Qt}, but having only one interface for three platforms makes this drawback worth having.


\subsubsection{Main Window}
\paragraph{}
The main window contains the \textit{OpenGL} scene view, the menu bar (on Mac OS X the menu bar is at the top of the screen) and a number of detachable widgets.
The scene view shows the look of the currently executing scene.
The widgets allow changing the scene or give more information.

\subsubsection{Design}
\paragraph{}
The design widget shows the scene graph as tree view.
A tree view is a standard widget, a list with items that can be expanded into sub-objects.
It is bound to the scene graph by a helper class that translates the calls from the widget to the backend scene graph.

The tree view item model uses placeholder items for each item.
Those place-holder items contain pointers to the represented object, it is one of the few places where raw pointers are used.
As a consequence it is also the place where most of the crashes during development happened.
For a detailed description of the api read the documentation on \lstinline{QAbstractItemModel} and \lstinline{QTreeView}.

To keep the tree-view in sync with the scene graph, the notification system of objects is used.
Every object notifies it's containing scene before and after external parameters change.
Container objects have special events for changes of the hierarchy such as adding or removing objects.
The scene in turn notifies listeners of changes in the layout of the scene.

\paragraph{}
The lower part of the design widget is filled by the settings pannel of the currently selected object.
Settings panels are instantiated by the object they belong to and get updated by the object when the object changes.
This is done by registering the settings pannel to receive change notifications form the object, similar to the method used to keep the tree view in sync.

Every object subclass has it's own corresponding parameter window subclass.
Those subclasses extend the parent's parameter window and add their own controls.

A screenshot containing the tree view and the parameter dialog of a mesh object can be seen in the appendix as figure \ref{ssTree}.


\subsubsection{Code}
\paragraph{}
Changing the code of a scene is not possible in the \ER\ user interface.
While adding a text editor would not be hard, offering the comfort of a full featured text editor is unfeasible.
The code widget is used instead to manage the additional files loaded by the current scene.

A list widget shows the loaded files, new or existing files can be added, and loaded files can be removed.
Removing files does not change the state of the running lua interpreter.
Since lua code is executed in a global address space, it would be impossible to find out which part of the memory is written from which file.

A preview of the currently selected file is shown in the lower area of the widget.
Double clicking on an item opens it in an external editor, if the operating system supports it.


\subsubsection{Log}
\paragraph{}
The log widget is fairly simple, it consists of a single \lstinline{QTextEdit} instance that displays the log.
The events are received from the program, which allows the registering of a callback for log output notifications.
A screenshot of the log in action can be seen as figure \ref{ssLog} in the appendix.

A more advanced implementation could allow filtering or date parsing, but at the current stage of the project such additional features do not seem needed.

