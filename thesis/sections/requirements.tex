\section{Requirements\label{Requirements}}
\paragraph{}
At the beginning of creating a tool should always be the requirements analysis.
Knowing what is needed, which problems are to be solved and what solutions already exists is essential.

In the following sections the currently existing solution is analyzed.
Then a few selected use cases are described.
Finally some requested features are listed.

\subsection{Existing Solutions}
\paragraph{}
Testing human subjects is a common practice in applied psychology.
One of the applications used for conducting tests is \textit{Super Lab}\cite{superlab}, a specialized stimulus presentation software.

It's main focus is ease-of-use for test creation.
It displays still-images, text and movies in a predefined or random order.
The detail control with the macro system is quite high for experienced users.
\textit{Super Lab} is specifically made for standard psychological user tests, so it's integration of of the notion of blocks and tests is strong.
The creation of tests is not simplified, the designer has to prepare complete pictures of each stimulus.

For tests in stereo vision, \textit{Super Lab} lacks some important features:
Output to projectors or in anaglyph is hard to achieve, animations and interactivity is limited and the creation of stimuli requires a lot of external work.
It's focus on some test types makes in unusable for others.

It's user-friendliness is high due to the limitation on only a small number of test types and the well known user interface.
The range input methods supported by \textit{Super Lab} is big, it ranges from keyboard and mouse input over specialized keyboards to voice input and eye trackers.


\subsection{Use Cases}
\paragraph{}
Before writing a program, one has to understand for what the program is to be used. The following examples illustrate the tests for which this framework has to be able to be used.

\begin{description}
\item[Focus]
The most basic test is to validate if a participant can focus and fuse on a target at an arbitrary location in a stereoscopic scene.
This is important in many other tests, where focus and binocular fusion has to be ensured.

\item[Fusion limit]
This test is used to ``test out'' the limits of fusion of a subject.
Knowing the limits, an experiment can be changed or subjects can be classified or rejected.

\item[Fatigue \& Continuous depth]
In this test two targets are shown resting on a ground plane, at three depth locations.
The targets show information that can only be decoded under fusion.
Measured is the reaction time of the test subject.
Before the test, after the test, and in the middle of the test a questionnaire is showed.
Difficulties include the complicated randomisation rule and the questionnaire.
For more details see section \ref{ExampleFatiguePilot}.

\item[Pinning]
Pinning is a border violation problem in stereoscopic images.
Testing it requires displaying various shapes and objects at locations near the border, in front or behind the parallax plane.
Texture can heavily influence how pinning is perceived.
Moving and rotating the objects can further help understanding the problem.

\item[Motion Sickness]
Testing the effect of motion in stereoscopic movies is important for understanding the creation of a pleasant or exciting viewing experience.
Simulating and testing requires the ability to interactively move both camera, and objects in a scene.
Motion should also be pre-definable, possibly as formula or as smooth movement along a curve.

\item[Testing Scenes]
When designing an experiment, being able to interactively change details of the scene, and get immediate feedback how it looks is invaluable
Scenes that are not used to test subjects, but to test the testing procedure do not require logging, instead they need interactivity and real time control.
\end{description}


\subsection{Features}
\paragraph{}
To implement the use cases in the previous sections, the framework has to contain high level functionality, some of which is listed here.

\begin{description}
\item[User interaction]
The test subject and what it does is the main interest of studies.
As such, the possibilities of interactions should not be arbitrarily constrained, but instead be as wide and flexible as possible.
The most basic interaction is by using the keyboard and the mouse.
In this group also fall devices that emulate them, such as joysticks track balls or touch sensitive screens.
More input vectors are imaginable, ranging from voice input and body movement to sensory data of the test subject.

\item[Logging]
Logging is the core of a test, it represents the data that the scientist is after.
As such, it has to be both exact and flexible.
A core requirement is that everything that happens, can be logged.
The log should be usable for any kind of test, and still be simple enough to understand and analyse.

\item[Survey]
While measurements of timing and correctness try to create an objective result, subjective feedback can not be neglected.
Popular in research are questionnaires, simple questions with answers on a given scale or in a given set of possible answers.

\item[Stereograms]
Stereograms encode depth information in a pair of 2D images.
So called auto-stereograms even manage to do this with one single image.
The depth information in a stereogram can only be extracted under binocular fusion, which makes it the perfect tool to ensure fusion.
Generating stereograms of various types and sizes is therefore a requirement in many tests.
See section \ref{Stereogram} for a detailed description of stereograms, and how they are implemented.

\item[Arbitrary object support]
Displaying a wide range of objects is required in many tests, simple planes, cubes and pictures are rarely enough.
But implementing all possible shapes is not possible, so loading already created shapes offers all the desired flexibility.
Texturing of imported objects is therefore an important feature.

\item[Camera and Object motion]
The camera and all objects have to be movable, both instantaneous and smooth.
Movement should be definable in many ways, for example as function, as path, or as interactive code that reacts on user input.

\item[Depth Cues]
Viewing depth relies not only on binocular stereopsis but also on a wide range of cues present in monocular images.
Those might be beneficial for a test, or influence the outcome in an undesired way.
Being able to enable and disable individual cues would be extremely useful.

\item[Randomisation \& Balancing]
Many parameters of a test have to be randomised, position of objects, color or shape of an image or the expected answer of a test.
But just randomising is not enough, often it is required to randomly permute a set of possibilities to create a balanced test.
In some cases even this is not enough, random orders have to be computed based on complex rules.

\item[Ease of use]
While not a singular feature, it is never the less very important that a product is easy to use.
If a product is too complicated to be used, it is useless, no matter how powerful it is.
The usability of a test is in the responsibility of the test designer, but making the creation of the test as simple and intuitive as possible is the job of the tool.
Increasing the usability ranges from creating a simple consistent user interface that offers basic functionality over adding powerful and extensible scripting capabilities for power user, to writing good documentation.
\end{description}
