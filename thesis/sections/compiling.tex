\section{Compiling \& Setup}
\paragraph{}
This section is only of relevance for developers who desire to change and further develop the system. Several steps have to be taken to compile the code successfully.
The details depend on the selected operating system, but the general ideas are similar for all three supported platforms.

\subsection{Prerequisites}
\paragraph{}
As described in section \ref{frameworks}, the tool is not written from scratch, it depends on a number of frameworks for user interface, Drawing and more.

OpenGL should be installed by default in all major operating systems. It is required to compile and run \ER.

A C++ compiler is required to compile the code.
On \textit{Mac OS X}, XCode and GCC are the recommended tools.
On \textit{Windows}, compiling with MinGW was tested with a precompiled Qt build.
The Qt binary installer for windows can install MinGW as well.
On \textit{Linux}, GCC was tested.

Qt from Trolltech can be downloaded for free in the GPL version\cite{qt}.
It is available for all three platforms either as source or precompiled.
Both the source and the binary versions were tested in version 4.4, but newer versions are expected to work as well.

Boost is a C++ source library intending to be compatible with the standard library, trying to eventually become part of the standard\cite{boost}.
As source library, it does not need to be compiled, just placed at a location where it is found by the compiler and build system.
A convenient way to do this is to extract it as directory \texttt{./examinationroom/boost}, the project file asumes this location.

All remaining libraries, such as LUA and LibObj are part of the source code, and do not have to be installed or compiled separately.


\subsection{Getting \ER}
\paragraph{Repository}
Development is managed with \textit{Git}\cite{git}, a distributed versioning system.
It is recommended to use the repository to stay up to date with development.
Git does not depend on a central server, local changes can easily be performed without impacting other developers.

The following commands will create a local repository from the master repository into the given folder and show the history.
Which repository is chosen as master depends on several factors such as the need for write access.
After the cloning, more repositories can be added.

\begin{verbatim}
git clone <repo-url> <folder>
cd <folder>
git log
\end{verbatim}

\paragraph{Project Page}
An archived version of pre-compiled binaries for Mac OS X and Windows is available on the project page\cite{project} (see screenshot \ref{ssProject}).
The binaries include all needed libraries, some demo files and lua libraries.
The project page also contains a link to the public read-only git repository, and the complete source code as zipped archive.


\subsection{Compiling}
\paragraph{}
After installing the libraries and getting the sources the next step is compiling the program.
This is a two step process, first build rules are generated from the Qt project file, and then, depending on the rule format chosen, the code is compiled.

\paragraph{XCode}
On \textit{Mac OS X}, the easiest and best method is to use XCode.
The shellscript \texttt{createXProj.sh} creates project files that contain all information to compile and run the application.
This process only has to be repeated when the project file changes.

\paragraph{MinGW}
On Windows, the free compiler MinGW was tested.
Executing \texttt{qmake} creates a make file that can be built with the tool-chain by invoking \texttt{make}.

\paragraph{Visual Studio}
Microsofts Visual Studio can compile \ER, sometimes.
A visual studio project is created with the file \texttt{createVS.bat}.
A visual studio makefile is created with \texttt{createWin32.bat}.
Note that the project structure is not preserved, and that the Visual Studio project file has to be regenerated every time the Qt Project file is changed.

\paragraph{Linux}
A makefile is generated with \texttt{qmake}, and compiled with \texttt{make}.


\subsection{Distribution}
\paragraph{}
Preparing a compiled ExaminationRoom binary for distribution requires some care.
The Qt library has to be bundled in a way the application can find it.
The following sections describe the approach to package a compiled executable for distribution.
For details and further reading, see the Qt deployment guide\cite{deployment}.

\subsubsection{Mac OS X}
\paragraph{}
On Mac OS X the recommended method for distribution is with the Qt Frameworks as private embedded framework.
All remaining frameworks are compiled into the application.
The following steps describe how to create such a binary.
For detailed commands and examples see the mac page on the Qt Deployment guide\cite{deployment}.

\begin{enumerate}
\item Make sure the installed Qt Framework is compiled as \textit{Universal Binary}, a code file that contains both PowerPC and x86 bytecode. This is the case in the pre-compiled distribution.
\item Build ExaminationRoom as \textit{Universal Binary} in release mode.
Make sure to disable all debug code beforehand.
\item To reduce the executable size, remove unused symbols from the code with the \texttt{strip} utility.
The compiled executable can be found inside the application bundle.
\item Copy the three Qt frameworks into the application bundle into \texttt{Contents/Frameworks/}.
ExaminationRoom uses QtCore, QtGUI and QtOpenGL.
\item If you compiled Qt yourself, it might be needed to change the framework identifier with \texttt{install\_name\_tool -id}.
This is not needed with the binary distribution, since it does not contain an absolute path.
\item Change the paths to the libraries in the ExaminationRoom executable.
The script changeFrameworkRef.sh can do this for the libraries installed into \texttt{/Library/Frameworks}, the location used by the binary distribution.
If the framework paths differ, use \texttt{install\_name\_tool -change} directly.
\texttt{otool -L} can list the paths to libraries referred to by code.
\item Do the same to the Frameworks.
The QtGUI and QtOpenGL framework refer to Core, and the QtOpenGL framework also links to QtGUI.
\item Copy the desired image format plugins into \texttt{Contents/plugins/}.
Make sure to preserve the folder structure.
\item Change the framework paths of the plugins as before, either with the script or manually.
\item Create the empty file at the location \texttt{Contents/Resources/qt.conf}.
The existence of this file causes Qt to search plugins at the default path instead of the hard coded path.
\item Bundle the Application with the res folder and the documentation. The recommended format is as Disk Image
\end{enumerate}

\subsubsection{Windows}
\paragraph{}
On Windows, the recommended method of distribution is with the Qt Libraries and plugins as dynamically linked files in the same directory.
While this does not provide the elegant single-object appearance as the mac version, it is easier to deploy.
The alternative is using static binding, but this requires a custom compiled Qt installation.
Depending on the used compiler, additional libraries have to be added.

\begin{itemize}
\item Build ExaminationRoom in release mode.
Make sure to disable all debug code beforehand.
\item Copy the executable into an empty folder
\item Copy the Qt libraries into this folder.
The following files are needed: \texttt{QtCore4.dll}, \texttt{QtGui4.dll}, \texttt{QtOpenGL4.dll}.
\item Copy the compiler library into this folder.
MinGW uses the file \texttt{mingwm10.dll}, Trolltech's page lists the dlls used in various Visual Studio compilers.
\end{itemize}
