\section{Compiling \& Setup}
\paragraph{}
This section is only of relevance for developers who desire to change and further develop the system. Several steps have to be taken to compile the code successfully.
The details depend on the selected operating system, but the general ideas are similar for all three supported platforms.

\subsection{Prerequisites}
\paragraph{}
As described in section \ref{frameworks}, the tool is not written from scratch, it depends on a number of frameworks for user interface, Drawing and more.

OpenGL should be installed by default in all major operating systems. It is required to compile and run \ER.

A C++ compiler is required to compile the code.
On \textit{Mac OS X}, XCode and GCC are the recommended tools.
On \textit{Windows}, Ming were tested with a precompiled QT build of the same compiler.
The QT binary installer for windows can install Ming as well.
On \textit{Linux}, GCC was tested.

QT from Trolltech can be downloaded for free in the GPL version\cite{qt}.
It is available for all three platforms either as source or precompiled.
Both the source and the binary versions were tested in version 4.3.4, but never versions are expected to work as well.

Boost is a C++ source library intending to be compatible with the standard library, trying to eventually become part of the standard\cite{boost}.
As source library, it does not need to be compiled, just placed at a location where it is found by the compiler and build system.
A convenient way to do this is to extract it as directory \texttt{./examinationroom/boost}, the project file asumes this location.

All remaining libraries, such as LUA and LibObj are part of the source code, and do not have to be installed or compiled separately.


\subsection{Getting \ER}
\paragraph{}
Development is managed with \textit{Git}\cite{git}, a distributed versioning system.
It is recommended to use the repository to stay up to date with development.
Git does not depend on a central server, local changes can easily be performed without impacting other developers.

The following commands will create a local repository from the master repository into the given folder and show the history.
Which repository is chosen as master depends on several factors such as the need for write access.
After the cloning, more repositories can be added.

\begin{verbatim}
git clone <repo-url> <folder>
cd <folder>
git log
\end{verbatim}


\subsection{Compiling}
\paragraph{}
After installing the libraries and getting the sources the next step is compiling the program.
This is a two step process, first build rules are generated from the QT project file, and then, depending on the rule format chosen, the code is compiled.

\paragraph{XCode}
On \textit{Mac OS X}, the easiest and best method is to use XCode.
The shellscript \texttt{createXProj.sh} creates project files that contain all information to compile and run the application.
This process only has to be repeated when the project file changes.

\paragraph{Ming}
On Windows, the free compiler Ming was tested.
Executing \texttt{createWin32.bat} creates a make file that can be built with the tool-chain by invoking \texttt{make}.

\paragraph{Visual Studio}
Microsofts Visual Studio can compile \ER, sometimes.
A visual studio project is created with the file \texttt{createVS.bat}.
Note that the project structure is not preserved, and that the Visual Studio project file has to be regenerated every time the QT Project file is changed.

\paragraph{Linux}
A makefile is generated with \texttt{qmake}, and compiled with \texttt{make}.
