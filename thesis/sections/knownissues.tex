\section{Known Issues \& possible improvements\label{Issues}}
\paragraph{}
While a great amount of time was spent on building and improving \ER, it is by no means complete.
Some features are still lacking, and creating a scene also requires understanding of programming in the language \textit{lua}.
This section contains a few selected possible improvements.

\paragraph{Usability}
The usability of \ER\ could be improved in several ways.
Creating the visual component of a scene is not as intuitive as it could be.
Clicking on objects to select them, or visual feedback for the selected object might be helpful.
Completely separating creation and execution of tests would allow this.

\paragraph{Programming}
Planing and designing a successful test is learnt over years in experimental psychology.
Writing programs is not.
And while the only way to achieve the power to conduct interactive experiments is a full programming language, most experiments do not require this power.

A big improvement in terms of usability would be to add a subclass of \lstinline{Program}, that does not rely on \textit{lua} but a simpler, easier to understand machine.
It could be implemented as a state machine(see section \ref{stateAPI}), or similar to \textit{Super Lab}\cite{superlab}.

\paragraph{Anaglyph}
During development, anaglyph rendering was only used for checking the output on a simple screen.
The current algorithm is sufficient for this.
The tests are conducted with projectors or a special screen.

In the future, testing anaglyph algorithms could be a possible task.
For this, a more flexible color mixing is needed.
The most promising approach was using \lstinline{GL_COLOR_MATRIX} with aux buffers.
It's drawback is the lack of speed and the problem with mixing.
Using \lstinline{pbuffer} objects or vertex shaders might be also worth a thought if backwards compatibility is not an issue.

\paragraph{Refactoring}
The design of \ER\ was always in flux, as new features were requested or problems encountered, it had to be changed to accomodate the new situation.
Simplifying this structure could lead to easier extensibility.

One candidate could be the container class.
Almost all containers only execute custom code before and after their contained objects are drawn.
Refactoring that out in virtual methods would make subclassing easier and lead to less duplicate code.

\paragraph{Object Access}
Currently, access to objects from lua is done through references, not by name.
Changing access to be by-name or to an other kind of indirect access would allow seamless replacement of the scene graph and undo handling.

This could for example be done by changing the \textit{lua} engine, or by adding new api calls to get volatile pointers by name, instead of storing persistent pointers in variables.

