\section{Coding Style}
\paragraph{}
How to structure programs, layout code or comment influences it's readability and maintainability.
The exact style does not influence this that much, but having a consistent style over a whole project is necessary.

When ever possible in \ER, the style described here is used.
Foreign components don't follow all the guidelines.

\subsection{Code file layout}
\paragraph{}
The source code of \ER\ is organized in a folder structure due to the big number of files.
It is advisable to follow the structure when adding new files.

\begin{description}
\item[{core}] contains the mechanic part of the program, with all object subclasses in \texttt{objects} and all texture subclasses in \texttt{surfaces}.
\item[{lua}] contains the lua library code and the luabridge library code.
\item[{parser}] contains the \textit{.obj} object file format parser library.
Additional formats should be added here.
\item[{proxy}] contains the custom binding code between the core and lua.
\item[{tools}] contains utility code that is of use in other parts of the program, among others platform abstraction code, a vector type and lua helpers.
\item[{ui}] contains the User Interface code.
The subfolder \texttt{parameter} contains the dialog boxes to change object-, texture- and camera-parameters.
The subfolder \texttt{renderer} contains the rendering methods.
\end{description}

\subsection{Naming Convention}
\paragraph{}
The general guideline to chose names is as follows, compound names can be created with \texttt{CamelCase}.

\begin{itemize}
\item Classes should be nouns and begin with a uppercase letter.
\item Methods should begin with a verb and a lowercase letter.
\item Method parameters should begin with a lowercase letter.
\item Local variables should begin with a lowercase letter.
\item Member variables and static member variables should begin with a lowercase letter and end with an underscore ("\texttt{\_}").
\item A getter should have the same name as the variable it sets, without the underscore.
\item A setter should have the verb \texttt{set} followed by the name of the variable it sets in camel case style.
\end{itemize}

\begin{lstlisting}[caption={Naming example}]
class ExampleClass
{
public:
	// method
	void updateStatus(float delta);
	// getter
	int memberVariable() const;
	// setter
	void setMemberVariable(int newMemberVariable);
private:
	int memberVariable_;
}
\end{lstlisting}


\subsection{Source formating}

\paragraph{Indenting}
Indenting code makes the flow and structure of functions and method apparent.
The style used in \ER\ is based on the \href{http://en.wikipedia.org/wiki/Indent_style}{Allman style}.
Curly brackets should always be on their own line, code inside them is indented with tabs.

\begin{lstlisting}[caption={Indenting example}]
//for (int i=0; i < x; i++)
//while (x == y)
if (x == y)
{
    something();
    somethingelse();
}
\end{lstlisting}

The advantage of this style is the clear separation between blocks.
Matching braces are in the same column, making it easy to find pairs.
All this helps with refactoring and understanding the concept of the code.


\paragraph{Commenting}
\ER\ uses \textit{Doxygen} style comments for all public methods in the header file of a class.
In complex methods, the code should be commented with C++ style comments ("\texttt{//}").

Doxygen comments are automatically extracted to generate API documentation.
It should be written keeping this in mind:
Code should not be referred to since the comments are often read without it, and context should be referred to by name and not location.

Comments should be written with the user in mind:
Point to similar functions, write what the method does, state it's side effects and describe the intended use.


\paragraph{Headers}
When ever possible, no external files should be included from a header file.
This benefits compile time, since changing a header file causes recompilation of all files that include it.

Ideally, only parents of classes and value members are included in a header file.
All other classes and types should be forward declared when ever possible and only included in code files if it is needed.
It is acceptable to include API header files, since they are not expected to change often.
\textit{C++}'s \texttt{using} statement should not be used in a header file.

\subsection{Other guidelines}
\paragraph{}
Those tips focus on the actual implementation and not only the look of the code.
It is advised to follow them to get a homogenous look-and-feel in \ER's source code.

\begin{itemize}
\item When ever possible, mark methods that don't change a class as constant.
\item When ever possible, use \lstinline{std::tr1::shared_ptr<Type>} instead of raw pointers.
\item When possible, pass data with a smart pointer or as const reference and not by value to reduce copy overhead.
\item Do not hardcode constants that make sense to be changeable, use static class members instead.
\item Comment out unused parameters so no unneeded warnings are emitted.
\item When committing code, add meaningful commit messages, describe what was changed.
\end{itemize}

