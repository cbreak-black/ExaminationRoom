\section{LUA API\label{LUAAPI}}
\paragraph{}
The LUA API is the interface with which the experiments are controlled from LUA.
LUA is the embedded scripting language used in the Tool.

The API of the framework is strongly based on the C++ API, almost all methods correspond directly to it.
The lua manual\cite{lua} describes the language and its constructs and standard libraries.
This chapter gives an introduction to the ideas behind the API, and then details some of the most important classes and methods.
For more details, refer to the C++ Doxygen documentation.


\subsection{LUA Crash Course}
\paragraph{}
The lua language is simple, but to understand its details, reading the manual\cite{lua} is recommended.
Here the basic constructs:

\subsubsection*{Expressions}

\begin{lstlisting}[language=lua]
x,y = math.cos(a)*r, math.sin(a)*r
\end{lstlisting}

\subsubsection*{Method call}
\begin{lstlisting}[language=lua]
Scene:addObject(Sphere())
\end{lstlisting}

\paragraph{}
The method \lstinline[language=lua]{addObject()} is called on the object \lstinline[language=lua]{Scene}.
Its arguments are the return values of the function call \lstinline[language=lua]{Sphere()}, which is the constructor for spheres.
This line of code creates a new sphere and adds it to the scene.
The sphere can not be moved or otherwise changed from lua anymore since the reference was not stored.

\subsubsection*{Table use}
\begin{lstlisting}[language=lua]
local t = {
    [1] = "First item";
    [2] = 2;
    ["three"] = x;
}
t.three = 3;
\end{lstlisting}

\paragraph{}
LUA tables support number indexes like a sparse array, and key based lookup like a map.
With string indexes tables can be treated like accessing structs in C.
The convention is to use 1 based arrays.
Tables are used in place of Vec vector objects when passing values to the API.

\subsubsection*{for loop}
\begin{lstlisting}[language=lua]
for i = 1,10,2 do
    Scene:log("Number :"..i);
end
for idx,obj in
    ipairs(objectArray) do
    Scene:log(obj:name());
end
\end{lstlisting}

\paragraph{}
For loops can iterate over a number sequence or over the content of a table.

\subsubsection*{function declaration}
\begin{lstlisting}[language=lua]
isPrime = function (num, div)
    if num%div == 0 then
        return false, div;
    else
        if divisor >
            math.sqrt(num) then
            return true;
        else
            return isPrime(num,
                div+1);
        end
end;
\end{lstlisting}

\paragraph{}
Neither type, nor amount of parameters or return values is declared.
A function is flexible in what can deliver, but it has to be flexible to deal with what it gets.
For example some parameters could be \lstinline[language=lua]{nil} or of a different type than expected.
Unused parameters are treated as \lstinline[language=lua]{nil}.
Functions are first class objects.
They can be referenced with variables, but also passed around.


\subsection{General Ideas}
\paragraph{}
Scenes can be built, animated and controlled from lua.
A scene is built by setting up a scene graph, and controlled by adding event handlers that change the scene graph.

\subsubsection{Building}
\paragraph{}
The visual part of a scene is a tree of objects.
Objects have a visual representation that is defined by their internal state.
Container objects can also influence how everything inside them is drawn (See \ref{sceneRep}).
Building a hierarchy of objects is therefore not done based on object relationship,
but on the modifiers that should be applied to them.

Scenes are built by instantiating objects, setting their parameters, and adding them to a scene hierarchy.
Objects can later be removed or moved inside the hierarchy.
Textures and Cameras are not subclasses of \lstinline{Object} and can not be directly added to a scene.
Instead they are added to objects that support them:
Textures can be added to most objects, cameras only to the scene and camera nodes.

\subsubsection{Animating \& Reacting}
\paragraph{}
In LUA there is no continuous flow of control, no run loop. Instead the code only runs when it is loaded and after that to handle events.
Four types of events are currently available: Key input can be received for both key-down and key-up. An update event is fired every time a new frame is about to be drawn. And finally, a quit event is fired when the application is about to be terminated.

\subsubsection{Differences from C++}
\paragraph{}
The main difference between the C++ and the corresponding LUA methods is for Vec parameters.
The Vec family of union structs represents 2, 3 or 4 element vectors.
In lua, the contained values are simply passed as lua tables of a suitable size.
Tables are one based.

Since lua allows more than one return value, some of the accessors could be simplified, a single method zoom could return both x and y values instead of a Vec2 object.
This is not done, but instead a table containing both values is returned.
This follows the convention of returning Vec parameters as tables.

Enumerators do not exist in LUA, string constants are used instead.

\subsection{The Scene}
\paragraph{}
The scene object exists once in every LUA context, accessible by the global variable \lstinline[language=lua]{Scene}. It should not be overwritten.
The main function of this class is to serve as representation of the current Scene Object instance.
It also offers some global camera control and most importantly event triggers.
Its C++ proxy class is \lstinline{LuaProxy}.

\subsubsection{Adding and removing}
\paragraph{}
Like all containers, the scene has the methods \lstinline[language=lua]{addObject(<Object>)} and \lstinline[language=lua]{removeObject(<Object>)} to add or remove objects respectively.
\lstinline[language=lua]{clearScene()} removes all objects from the scene.

\subsubsection{Camera Control}
\paragraph{}
The camera of a scene can be accessed directly with \lstinline[language=lua]{camera()}, and replaced with other camera instances through \lstinline[language=lua]{setCamera(<Camera>)}.
For legacy reasons, there also exist direct camera modifiers.

\subsubsection{Logging}
\paragraph{}
Logging is important in many tests. To give easy access to a log file, combined with high accuracy timestamps, this is done in C++.
The scene provides functions to write those messages to the log file with \lstinline[language=lua]{log(<String>)} or the standard output with \lstinline[language=lua]{debugLog(<String>)}.

\subsubsection{Events}
\paragraph{}
To register event handlers, the method \lstinline[language=lua]{setEventListener(<String>, <Function>)} is used.
The strings \lstinline[language=lua]{"update"}, \lstinline[language=lua]{"keyDown"} and \lstinline[language=lua]{"keyUp"} are valid, and correspond with the associated events.

The update event function gets the time in seconds that passed since the last invocation of the update event as argument.
The key down and key up functions get the pressed or released key as string as argument.
The quit function gets zero as argument.

\subsection{The Objects}
\paragraph{}
The LUA bindings are implemented with the template based library \textit{luabridge}\cite{lb}.
It can automatically generate api wrapper code for most basic types of functions with up to 8 parameters.
Objects are registered in the method \lstinline{Program::registerComponents()},
by calling each object's \lstinline{registerLuaApi()} method.
Subclasses can overwrite this method and register their own methods.

\subsubsection{Creation}
\paragraph{}
The object constructors are global functions with the same name as the class.
The following classes are provided with the program:
\lstinline[language=lua]{Sphere()}, \lstinline[language=lua]{Rectangle()}, \lstinline[language=lua]{Parallelepiped()}, \lstinline[language=lua]{Pixelplane()}, \lstinline[language=lua]{Text()}, \lstinline[language=lua]{Mesh()}, \lstinline[language=lua]{AffineTransformation()}, \lstinline[language=lua]{CameraNode()}, \lstinline[language=lua]{LightNode()}, \lstinline[language=lua]{Atmosphere()}, \lstinline[language=lua]{DepthBuffer()}, each corresponding with the class with the same name.

\subsubsection{Base}
\paragraph{}
Every object has a position, a color, a draw priority, visibility and a few other attributes.
Attributes can be queried like \lstinline[language=lua]{<attribute>()} and changed like \lstinline[language=lua]{set<Attribute>(<newValue>)}.

\subsubsection{Container}
\paragraph{}
Containers have the purpose of applying something to everything they contain.
Like the scene object, they support adding and removing objects with \lstinline[language=lua]{addObject(<Object>)} and \lstinline[language=lua]{removeObject(<Object>)}.
The functionality of containers can be enabled independently of the visibility of their sub objects with \lstinline[language=lua]{setEnabled(<bool>)}

\subsection{The Rest}
\subsubsection{Camera}
\paragraph{}
Cameras represent a transformation from 3D to 2D space.
They can use perspective projection, parallel projection and screen space projection, which is chosen with \lstinline[language=lua]{setType(<String>)}, valid strings are \lstinline[language=lua]{"Perspective"}, \lstinline[language=lua]{"Parallel"} and \lstinline[language=lua]{"Screen"}.

The other camera properties can also be changed with accessors.

\subsubsection{Texture}
\paragraph{}
Textures are created by loading image data from a file.
Like objects, they are bound by \textit{luabridge}, and follow the same rules.
The constructors are global functions with the name of the class, one of the following:
\lstinline[language=lua]{Texture()}, \lstinline[language=lua]{Stereogram()}, \lstinline[language=lua]{RandomDot()} or \lstinline[language=lua]{Pattern()}.

Simple textures take one argument, the string to an image file.
Stereogram textures take two image paths as argument, and load the left and right eye view directly.
RandomDot textures take one path as argument, and use this as depth map to construct a random dot stereogram.
Pattern textures work similar, but they take a second and optional third argument containing the path to a pattern for background and foreground.

