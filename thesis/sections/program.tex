\subsection{Program}
\paragraph{}
The class \lstinline{Program} encapsulates a test scene.
It contains the scene graph (see \ref{SceneGraph}) and the scene mechanics (see \ref{Mechanic}).

The program mediates user input and relays it to the engine
and handles loading and storing of a scene.

The scene also manages names.
Names are important especially for serialisation, since lua variables are referred to by name.
Uniqueness and validity is ensured, and reserved lua keywords are protected.

\paragraph{Internal interaction}
In the current implementation, the scene mechanic code accesses the scene graph directly,
and has references to the actual objects.
This makes modularisation more difficult, and makes it impossible to replace a whole scene graph with a duplicate.
An improvement would be to only refer to objects by name or with some sort of \textit{URI}.
Implementing such a scheme would require a huge effort especially due to the nature of the binding to \textit{lua}.
\textit{LuaBridge} does not support polymorphism.

