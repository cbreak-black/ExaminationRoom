\section{Log Parser\label{Log}}
\paragraph{}
\ER\ generates  a log file of all tests containing time-stamped events.
For most tests this format is not easily usable.
The log parser application \textit{Statisticsroom} transforms those log files into a format that can easily be analyzed.
A screenshot can be seen as figure \ref{ssTransform} in the appendix.

The parser works by recognizing individual test stimuli and responses.
When the stimulus start entry in the log is recognized, all following recognized log lines are used to atribute data to this stimulus.
When the stimulus end log entry is found, the time between start and end is calculated and written together with the attributed data fields.

\subsection{Log format}
\paragraph{}
The log format of \ER\ is very simple.
The time stamp format is \lstinline{"yyyy.MM.dd hh:mm:ss.zzz"} following the ISO date standard.
The year is written with four numbers, month and day are printed with two numbers including leading zeros, all separated by dots.
The time follows separated by a space, with hour, minute and seconds in 24 hour format, separated by colons, milliseconds by a point.

The log message is not constrained.
A good log message however contains strings that make it identifiable, and values that cary data.
This is important to be able to write regular expressions that recognize special types of log lines.


\subsection{Output Format}
\paragraph{}
The product of the transformation is a list of values grouped by stimulus.
The values include the stimulus number and the stimulus duration, as well as all associated values.
This is written to the output file as tab separated values (CSV, comma separated values).

\paragraph{Excel}
\textit{Microsoft}'s \textit{Excel} can import tab separated values.
It was the most used tool for data analysis during the conduction of the fatigue test (\ref{FatiguePilot}).
The pivot table function makes selecting, grouping and analyzing data fast and precise.
It easily groups lines with the same values, the same stimulus query, response type, correctness or similar together.

\paragraph{OpenOffice.org}
\textit{OpenOffice.org}, the open source spreadsheet application, can import CSV as well.
It is not used as much as the competition, but it's power is comparable with that of \textit{Excel}'s level.

\paragraph{MySQL}
CSV data can easily be read into \textit{MySQL} with the \texttt{LOAD DATA INFILE} command or the \texttt{mysqlimport} command line tool.
The filtering and access capabilities of SQL is superior with big data volumes than that of spreadsheet applications.


\subsection{Regular Expressions}
\paragraph{}
The transformation process identifies log lines of interest and extracts data from them.
The most flexible to do this is with regular expressions.
\textit{Qt} has support for regular expressions built in which simplifies the implementation.

A regular expression is a string that can be matched against other strings, it describes a class of strings.
The special character \texttt{\^} represents the start of the string, \texttt{\$} represents it's end.
The dot \texttt{.} matches every character, \texttt{\textbackslash d} a digit, \texttt{\textbackslash s} a white-space and \texttt{\textbackslash w} a word-character.

Quantifiers match a repetition of the previous item.
They are greedy, and try to match as many occurrences as possible.
The star \texttt{*} matches zero or more repetitions.
The plus sign \texttt{+} matches one or more repetitions.
The question mark \texttt{?} matches zero or one repetition.

More information on this topic can be found on \textit{Qt}'s documentation page on \lstinline{QRegExp}\cite{regexp}.

\paragraph{Captures}
A capture is a special part of a regular expression that gets extracted if the expression matches.
Captures can be used internally in an expression, but it's primary use is to get data out of the log file.
Expressions without capture just count the number of matches, but expressions with capture return the capture in those matches.

\verb|^\((-?\d+\.\d+), (-?\d+\.\d+)\)$| for example matches \texttt{(1.2, -3.4)} and returns the numbers $1.2$ and $-3.4$ as text.

\paragraph{Output Formating}
For every regular expression, one or more titles can be added separated by semicolons.
For every title, one capture is printed and one column is built.
An exception are the start and end capture, which together form a single column with the duration of the stimulus as content.

