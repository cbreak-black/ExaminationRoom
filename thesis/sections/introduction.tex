\section{Introduction\label{Introduction}}
\paragraph{}
Researching the effect of the viewing of stereoscopic stimuli requires a different tool than usual user testing. The stimuli have to be more complex, and more dynamic. While traditional tests only require simple input and don't need to react on it, changing the stimuli based on what happened is often a requirement for the tests planed in stereoscopic vision research.

The following sections will introduce several examples of use cases, tests in which the framework should enhance the procedures and lessen the work done in test design and execution.
In the features section, some important abilities needed in those tests are detailed.

\subsection{Use Cases}
\paragraph{}
Before writing a program, one has to understand for what the program is to be used. The following examples illustrate the tests for which this framework has to be able to be used.

\paragraph{Focus test}
The most basic test is to validate if a participant can focus and fuse on a target at an arbitrary location in a stereoscopic scene.
This is important in many other tests, where focus and binocular fusion has to be ensured.

\paragraph{Fusion limit test}
This test is used to ``test out'' the limits of fusion of a subject.
Knowing the limits, an experiment can be changed, or subjects can be classified or rejected.

\paragraph{Fatigue and Continuous depth}
In this test two targets are shown resting on a ground plane, at three depth locations.
The targets show information that can only be decoded under fusion.
Measured is the reaction time of the test subject.
Before the test, after the test, and in the middle of the test a questionnaire is showed.
Difficulties include the complicated randomisation rule and the questionnaire.
For more details see section \ref{ExampleFatiguePilot}.

\paragraph{}
\todo{Write short paragraphs about the different tests}

\begin{itemize}
\item Pinning
\item Motion Sickness
\end{itemize}

\subsection{Features}
\paragraph{}
To implement the use cases in the previous sections, the framework has to contain high level functionality, some of which is listed here.

\paragraph{Logging}
Logging is the core of a test, it represents the data that the scientist is after.
As such, it has to be both exact and flexible.
A core requirement is that everything that happens, can be logged.
The log should be usable for any kind of test, and still be simple enough to understand and analyse.

\paragraph{Stereograms}
Stereograms encode depth information in a pair of 2D images.
So called auto-stereograms even manage to do this with one single image.
The depth information in a stereogram can only be extracted under binocular fusion, which makes it the perfect tool to ensure fusion.
Generating stereograms of various types and sizes is therefore a requirement in many tests.
See section \ref{Stereogram} for a detailed description of stereograms, and how they are implemented.

\paragraph{}
\todo{Write short paragraphs about what is needed for those tests}

\begin{itemize}
\item Complex randomisation
\item User interaction
\item Environment
\item Questionnaire
\item Constant Size
\item Complex Shapes
\item Complex camera motion
\end{itemize}
