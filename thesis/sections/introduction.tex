\section{Introduction\label{Introduction}}
\paragraph{}
Researching the effect of the viewing of stereoscopic stimuli requires a different tool than usual user testing.
While traditional tests only require simple input and often don't need to react on it, changing the stimuli based on what happened is often a requirement for the tests planned in stereoscopic vision research.
The stimuli have to be more complex than simple bitmap images as those in standard user tests.

Which features would simplify the creation and conduction of such tests?
How can a tool be flexible enough to support every imaginable type of test, while still be user friendly?
What is the right tradeoff between power and usability?

The following sections will introduce several examples of use cases, tests in which the framework should enhance the procedures and lessen the work done in test design and execution.
In the features section, some important abilities needed in those tests are detailed.


\subsection{Program Design}
\paragraph{}
\todo{Give overview over program design part}

\subsection{Implementation}
\paragraph{}
\todo{Give overview over implementation part}

\subsection{Documentation}
\paragraph{}
\todo{Give overview over documentation part}

