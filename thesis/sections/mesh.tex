\subsection{Meshes\label{Meshes}}
\paragraph{}
An important requirement for the tool is that it has to be able to display a wide range of stimuli.
Not only the depth cues (see section \ref{Cues}) should be individually controllable, but also the possible shapes of objects have to span a wide range.

The tool supports only few basic object types, such as Pixel Planes (raster image data), parallelograms, and parallelepipeds. Implementing every desired type in code is neither feasible nor desirable.
Instead, the same approach as with textures is taken: An external tool that is familiar to the user is used to create a shape, and this shape is then loaded and displayed by the tool. By far the most commonly used method to store models is \textit{Mesh}. A mesh is a surface defined by points, which are connected by edges, to form faces. Other methods include \textit{NURBS}, \textit{Constructive Solid Geometry} and many types of \textit{Parametric Surfaces}. They are less popular, and often harder to handle than meshes.
\todo{Find someone who says the same}

There are a number of possibilities to implement this design. A mesh library can be used to load, render and modify meshes in an easy way. A scene graph library provides not only mesh support, but also the handling of whole scene hierarchies. Mesh importer libraries make it easier to parse models into a custom data format.


\subsubsection{Mesh Libraries}
\paragraph{}
Mesh libraries provide an API for loading, manipulating and rendering meshes. \textit{OpenMesh} is a very powerful library with a half-edge data structure. It offers a wide array of possibilities to customize storage, access and manipulation. It does not handle rendering.

\paragraph{}
\todo{Write about OpenMesh and others}

\subsubsection{Mesh Formats}
\paragraph{}
\todo{Write about Obj and others}
\todo{Write about normal and texture coordinate data}

\subsubsection{Mesh Creation}
\paragraph{}
\todo{Write about Maya and others}