\subsection{Meshes\label{Meshes}}
\paragraph{}
An important requirement for the tool is that it has to be able to display a wide range of stimuli.
Not only the depth cues (see section \ref{Cues}) should be individually controllable, but also the possible shapes of objects have to span a wide range.

The tool supports only few basic object types, such as Pixel Planes (raster image data), parallelograms, and parallelepipeds. Implementing every desired type in code is neither feasible nor desirable.
Instead, the same approach as with textures is taken: An external tool that is familiar to the user is used to create a shape, and this shape is then loaded and displayed by the tool.
The most common and basic method to store models is as \textit{Polygon-Mesh} (see \cite{MeshPop} for an overview of common techniques).

A mesh is a surface defined by points, which are connected by edges, to form faces. Other methods include \textit{Subdivision Surfaces}, \textit{Constructive Solid Geometry} and many types of \textit{Parametric Surfaces}.
They are less popular and harder to handle than meshes.
\textit{OpenGL} can directly render some mesh data and has easy bindings for free form meshes.

There are a number of possibilities to support mesh loading and rendering.
A mesh library can be used to load, render and modify meshes in an easy way.
A complete scene graph library provides not only mesh support, but also the handling of whole scene hierarchies.
Mesh importer libraries make it easier to parse models into a custom data format.


\subsubsection{Mesh Libraries}
\paragraph{}
Mesh libraries provide an API for loading, manipulating and rendering meshes.
The final choice was \textit{libObj}, but several alternatives were considered.

\paragraph{OpenMesh}
\textit{OpenMesh} is a very powerful library with a half-edge data structure.
It offers a wide array of possibilities to customize storage, access and manipulation,
but it does not handle rendering.
The half-edge data structure is tuned for computations and easy traversal.
Most of it's features are not useful in \ER.

\paragraph{libMesh}
The \textit{libMesh} library is also more than just a mesh loader and data storage.
It is specially tuned to run numerical simulations of partial differential equations
It's data formats are focused on mathematical analysis which makes it of little use in \ER.

\paragraph{libObj}
The library \textit{libObj}\cite{libobj} is a simple parser library for \textit{Alias Wavefront}'s \textit{OBJ} file format.
It is currently supported by all major 3D applications as export format.
The \textit{OBJ} format contains vertexes and mesh information.
It optionally also contains normal and texture data per vertex.
Other features of the format are not supported.
This is the library that was finally used in \ER to load meshes.

Internally, an imported mesh is stored as list of triangle objects.
To speed up rendering, the mesh is also stored in an \textit{OpenGL} display list.

\subsubsection{Mesh Creation}
\paragraph{}
While the file format is in text form, creating meshes by hand in a text editor is not the recommended path.
Specialized tools for modeling make the task of creating good meshes easier.
Popular tools are \textit{Maya}, \textit{3D Studio Max} and \textit{Cinema 4D}.
A free and open source alternative is \textit{Blender}.

Independently of which tool is used, the exchange format between it and \ER\ is \textit{Alias Wavefront}'s \textit{OBJ} file format.
Normals and texture coordinates should be exported along with the mesh.
Currently only one texture per mesh can be assigned in \ER.


