\documentclass[11pt]{scrartcl}
%\documentclass[12pt, a4paper]{article}

\newif\ifpdf
\ifx\pdfoutput\undefined
\pdffalse % we are not running PDFLaTeX
\else
\pdfoutput=1 % we are running PDFLaTeX
\pdftrue
\fi

\ifpdf
\usepackage[pdftex]{graphicx}
\else
\usepackage{graphicx}
\fi

\usepackage[utf8]{inputenc}
\usepackage[english]{babel}
%\usepackage{german}
%\usepackage{longtable}
%\usepackage{tocbibind}
%\usepackage{makeidx}
\usepackage[pdftex,pageanchor,colorlinks,pdfborder=0,breaklinks,urlcolor=blue]{hyperref}
\usepackage{amsmath}
\usepackage{amsfonts}
\usepackage{amssymb}
%\usepackage{pdfsync}  % enable tex source and pdf output syncronicity
%\usepackage[all]{xy}
\usepackage{multicol}
%\usepackage{rotating}
%\usepackage{wrapfig}
%\usepackage{subfigure}
%\usepackage{listings}

\usepackage{geometry} % to change the page dimensions
\geometry{a4paper}
\geometry{textwidth=16.5cm,textheight=24.5cm}
\geometry{left=2.5cm,twoside}
\parskip=0 cm
\parindent=0.0cm

\usepackage{fancyhdr} % This should be set AFTER setting up the page geometry
\pagestyle{fancy} % options: empty , plain , fancy

\renewcommand{\labelitemi}{-}

% Tiefe des Inhaltsverzeichnisses
\setcounter{secnumdepth}{2}
\setcounter{tocdepth}{2}

\hypersetup{
    pdftitle={Master Thesis: Tool support for user testing of stereoscopic fatigue},
%    pdfsubject={Subject of the document}, % Subject 
    pdfauthor={Gerhard Roethlin},              % Author
    pdfkeywords={master, fatigue testing software, job description},       % Keywords
}

\title{Master Thesis: Tool support for user testing of stereoscopic fatigue\\
{\large Implementing a flexible and extensible testing framework}}
\author{\normalsize Gerhard R\"othlin 
{\tt  \href{mailto:gerhardr@student.ethz.ch}{gerhardr@student.ethz.ch}}}
\date{}

\ifpdf
\DeclareGraphicsExtensions{.pdf, .jpg, .tif}
\else
\DeclareGraphicsExtensions{.eps, .jpg}
\fi

\begin{document}

\maketitle
%\tableofcontents

%\begin{center}
%\includegraphics[width=16cm,clip,trim=0cm 0cm 0cm 0cm]{media/dollyCover.jpg}
%\end{center}

%\begin{abstract}
%\end{abstract}

\begin{multicols}{2}

%\tableofcontents

\section{General research questions}
\paragraph{}
The general research questions in this area are:
\begin{itemize}
\item What are the physiological and psychological limits of stereoscopic viewing fatigue?
\item How can stereo images be composed to reduce this fatigue?
\end{itemize}

\section{Example requirement}
\paragraph{}
One possible test is the following, but the tool should not be limited to this:

\begin{enumerate}
\item Encode information at a certain perceived stereoscopic depth. Measure the time it takes the test subject to addapt to this depth and extract the information.
\item Remove the information and place it at an other depth location, measuring reaction time.
\item Repeat step 1 and 2 at a different depth locations.
\item Repeat step 1 and 2 with the information being embedded in a scene.
\item Measure fatigue over a longer period of time, over several test repetitions.
\end{enumerate}

\section{Challenges}
\paragraph{}
To perform those tests, a flexible and extensible tool is needed. Those are some of the callenges that have to be overcome:

\begin{enumerate}
\item Supporting stereogram encoding of information.
\item Supporting of various scenes/depth clues
\item Flexible and powerfull log format, that can be transformed to be analyzed by established software
\item User input while looking at screen, such as blind typing, audio feedback
\item Must support various scenes and scenes changing in various ways
\end{enumerate}

\section{Baseline features}
\paragraph{}
The software should be created for specific tests, and then later extended to other areas. During development, a pilot study will be conducted. Having immediate feedback will help to guide development.

The tool should have at least those baseline featuers:

\begin{itemize}
\item Rendering in 3D
\item Import of 3D models and other data from other applications
\item Interactive modifications of scene in limited ways during the test
\item Must support anaglyph and dual screen/mirrored viewfinder output
\item Support of basic scene objects, individually enabled
\item Scripting of tests, with randomized parameters
\item Individual controll of depth cues, such as relative size, occlusion or stereoscopic seperation
\item Support various input mechanisms such as keyboard, mouse.
\item Record all events
\item Support exprort of logs in various formats
\end{itemize}

\section{Shopping list}
\paragraph{}

My thoughts about the design:
\begin{itemize}
\item It should be cross platform, so that it works on Mac OS X (my primary development platform), Windows and Linux.
\item It should support 3D rendering with HW acceleration, so OpenGL is the way to go
\item I recommend QT as GUI/Backend toolkit. I've made good experiences with it in my last project, dolly, and it is cross platform and licensed under GPL v2.
\item I recommend OpenMesh as mesh loader and storage library. (If we want mesh loading at all.) It is licensed under the GPL v2.
\item I recommend LUA (which I know from World of Warcraft) or some other scripting language to do the interactive stuff. It leaves {\textit a lot} of freedom to the designer of the scene/test, since most of the functionality would run in script code. LUA is licensed under an MIT style attribution license, which allows use everywhere.
\item Test files would contain media (pictures, models, textures), lua code (for building scenes out of the media, and animating, moving, randomizing) and a master file that binds it together.
\item We could provide templates or even a test builder application to create those lua scripts in standard cases, but writing code would give us {\textit a lot} of flexibility.
\end{itemize}

\end{multicols}

\appendix

\section{Links}
\url{http://graphics.cs.uiuc.edu/svn/wickbert/trunk/OpenMesh/LICENSE}

\url{http://trolltech.com/products/qt/licenses/licensing/opensource}

\url{http://www.lua.org/license.html#5}

\url{http://www.rasterbar.com/products/luabind/docs.html}

\url{http://lua-users.org/wiki/LunaWrapper}





\end{document}
\end
